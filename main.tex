\documentclass[aspectratio=169,xcolor=dvipsnames]{beamer}
\usepackage[utf8]{inputenc}  % Required for umlauts
\usepackage[english]{babel}  % Language
%\usepackage[sfdefault]{roboto}  % Enable sans serif font roboto
%\usepackage{libertine}  % Enable this on Windows to allow for microtype
\usepackage[T1]{fontenc}  % Required for output of umlauts in PDF

\usepackage{mathtools,bbold}  % Required for formulas
\usepackage{siunitx}  % Give numbers units (with proper spacing)

\usepackage{caption}  % Customize caption aesthetics
\usepackage{tcolorbox}  % Fancy colored boxes
\usepackage{color}
\usepackage{xcolor}  % Highlighting
\usepackage{soul}

\usepackage{booktabs}  % Using pandas' LaTeX output
\usepackage{multirow}  % Enable fancy table structure
\usepackage{listings}  % Insert programming code
\usepackage{lstautogobble}  % Cleaner indentation in TeX file of code blocks within LaTeX blocks

\usepackage{graphicx}  % Required to insert images
\usepackage{subcaption}  % Enable sub-figure
\usepackage[space]{grffile}  % Insert images baring a filename which contains spaces
\usepackage{float}  % Allow to forcefully set the location of an object
% Include external standalone files such as Tikz graphics; requires `-shell-escape`
\usepackage{standalone}
\usepackage{tikz}  % Fancy drawing environment
\usepackage{pgfplots}  % Functions in Tikz
\usepackage[]{algorithm2e}  % Format fancy algorithms

\usepackage[tracking=true]{microtype} % Required to change character spacing

\usepackage[backend=biber,autocite=footnote,style=authoryear-icomp,sorting=none,doi=false,isbn=false,url=false,eprint=false]{biblatex}
\usepackage{csquotes}  % Ensure proper quotation of texts with babel and polyglossia with biblatex
\usepackage{hyperref}  % Insert clickable references

\usepackage{datetime}  % Flexible date specification

\usepackage{geometry}
\usepackage{scrextend}  % Allow arbitrary indentation

\usepackage{appendixnumberbeamer}  % Fancy page numbering excluding the appendix

% Compile notes into a separate file readable by pdfpc using a custom package which overwrite the `note` macro
\usepackage{pdfpcnotes}
% NOTE: Adding "noframenumbering" as argument to the first slide confuses `pdfpc`; see https://github.com/pdfpc/pdfpc/issues/367

\usetikzlibrary{patterns}
\usetikzlibrary{arrows}
\usetikzlibrary{matrix}
\usetikzlibrary{hobby}
\usetikzlibrary{shapes.misc}
\usetikzlibrary{shapes.callouts}
\pgfmathdeclarefunction{gauss}{2}{%
	\pgfmathparse{1/(#2*sqrt(2*pi))*exp(-((x-#1)^2)/(2*#2^2))}%
}

\addbibresource{literature.bib}
\renewcommand{\footnotesize}{\tiny}
%\renewcommand*{\bibfont}{\scriptsize}

\newcommand{\leadingzero}[1]{\ifnum#1<10 0\the#1\else\the#1\fi}
\newcommand{\todayddmmyyyy}{\leadingzero{\day}.\leadingzero{\month}.\the\year}
\newcommand{\mathcolorbox}[2]{\colorbox{#1}{$\displaystyle #2$}}

\DeclareMathOperator*{\argmin}{arg\,min}

\makeatletter
% Fix subfig in beamer style presentation
\let\@@magyar@captionfix\relax

% Insert [short title] for \section in ToC
\patchcmd{\beamer@section}{{#2}{\the\c@page}}{{#1}{\the\c@page}}{}{}
% Insert [short title] for \section in Navigation
\patchcmd{\beamer@section}{{\the\c@section}{\secname}}{{\the\c@section}{#1}}{}{}
% Insert [short title] for \subsection in ToC
\patchcmd{\beamer@subsection}{{#2}{\the\c@page}}{{#1}{\the\c@page}}{}{}
% Insert [short title] for \subsection in Navigation
\patchcmd{\beamer@subsection}{{\the\c@subsection}{#2}}{{\the\c@subsection}{#1}}{}{}
\makeatother

\definecolor{dodgerblue}{rgb}{0.06, 0.44, 0.8}
\setbeamercolor{tableofcontents}{fg=dodgerblue}
\setbeamercolor{section in toc}{fg=black}
\setbeamercolor{subsection in toc}{fg=black}
\setbeamercolor{block title}{fg=black}
\setbeamercolor{qed symbol}{fg=black}
\setbeamercolor{enumerate item}{fg=black}
\setbeamercolor{itemize item}{fg=black}
\setbeamercolor{itemize subitem}{fg=black}
\setbeamercolor{title}{fg=dodgerblue}
\setbeamerfont{title}{size=\LARGE}
\setbeamertemplate{title page}{%
	\vbox{}
	\begin{centering}
		\begin{beamercolorbox}[sep=8pt,center]{title}
			\usebeamerfont{title}\inserttitle\par%
			\ifx\insertsubtitle\@empty%
			\else%
				\vspace{0.25em}
				{\usebeamerfont{subtitle}\usebeamercolor[fg]{subtitle}\insertsubtitle\par}%
			\fi%
		\end{beamercolorbox}%
		\vspace{1em}\par
		\begin{beamercolorbox}[sep=8pt,center]{author}
			\usebeamerfont{author}\insertauthor%
		\end{beamercolorbox}
		\begin{beamercolorbox}[sep=8pt,center]{institute}
			\usebeamerfont{institute}\insertinstitute%
		\end{beamercolorbox}
		\begin{beamercolorbox}[sep=8pt,center]{date}
			\usebeamerfont{date}\insertdate%
		\end{beamercolorbox}%\vskip0.5em
	\end{centering}
}
\setbeamercolor{frametitle}{fg=dodgerblue}
\setbeamerfont{frametitle}{size=\LARGE}
\setbeamerfont{framesubtitle}{size=\small}
\setbeamertemplate{frametitle}{%
	\nointerlineskip%
	\hspace*{0.45em}
	% Other decent options are: `center`
	\begin{beamercolorbox}[ht=4em,sep=1em,wd=\paperwidth]{frametitle}
		\usebeamerfont{framesubtitle}
		\hspace{-0.3em}\strut\insertframetitle\strut%
		\\
		\usebeamerfont{frametitle}
		\strut\insertframesubtitle\strut%
	\end{beamercolorbox}
}
\setbeamercolor{footline}{fg=gray}
\setbeamercolor{date in head/foot}{fg=gray}
\setbeamercolor{author in head/foot}{fg=gray}
\setbeamercolor{section in head/foot}{fg=gray}
\setbeamerfont{footline}{size=\tiny}
\setbeamertemplate{footline}[text line]{%
	\leavevmode%
	\hspace*{-3.2em}
	\hbox{%
		\begin{beamercolorbox}[wd=.33\paperwidth,ht=1em,dp=0.5em,left]{date in head/foot}%
			\hspace{1.5em}
			\usebeamerfont{date in head/foot}\insertshortdate
		\end{beamercolorbox}%
		\begin{beamercolorbox}[wd=.33\paperwidth,ht=1em,dp=0.5em,center]{author in head/foot}%
			\usebeamerfont{author in head/foot}\insertshortauthor
		\end{beamercolorbox}%
		\begin{beamercolorbox}[wd=.33\paperwidth,ht=1em,dp=0.5em,right]{section in head/foot}%
			\usebeamerfont{date in head/foot}
			\insertframenumber{} %/ \inserttotalframenumber\hspace*{1em}  % NOTE, excessive for a 12 min. talk
			\hspace{1.5em}
		\end{beamercolorbox}
	}
}
\setbeamertemplate{navigation symbols}{}
\setbeamertemplate{itemize item}{$\bullet$}
\setbeamertemplate{itemize subitem}{$\circ$}
\captionsetup{font=scriptsize,labelfont={bf,scriptsize}}

% Customize code blocks
\definecolor{dodgerblue}{rgb}{0.06, 0.44, 0.8}
\definecolor{dodgerred}{rgb}{0.93, 0.24, 0.26}
\lstset{basicstyle=\ttfamily,
	breaklines=true,
	showstringspaces=false,
	commentstyle=\color{dodgerred},
	keywordstyle=\color{dodgerblue},
	frame=none,
	frameround=ffff,
	autogobble=true
}
\lstset{language=Python,
	basicstyle=\ttfamily\scriptsize,
	rulecolor=\color{black},
	tabsize=2,
}

\title{NIFTy: The Why and How of Building AD from Scratch}
\subtitle{}
\author[Gordian Edenhofer]{%
	{\href{mailto:gordian.edenhofer@gmail.com}{Gordian Edenhofer}}\inst{1,2,3}
}
\institute[LMU]{%
	\inst{1}Max Planck Institute for Astrophysics, Garching \\
	\inst{2}Faculty of Physics, LMU, Munich \\
	\inst{3}Center for Astrophysics $\vert$ Harvard \& Smithsonian, Boston \\
}
\date[EnzymeCon]{Enzyme Conference 2023, \formatdate{23}{02}{2023}}
\subject{}

\begin{document}

% NOTES
% * Introduce myself!!!
% * Do NOT talk about TOC for a 12 min. talk

\pagenumbering{arabic}

\begin{frame}[plain]
	\titlepage%
	\note{%
		* From astrometric and photometric data to 3D dust
	}
\end{frame}

% Submitted Abstract
%
% Automatic Differentiation (AD) is the backbone of applied second order
% minimization schemes and used extensively for solving statistical inference
% problems. Both often require forward and reverse mode differentiation for
% efficiency. However, early AD frameworks did not support both. In 2013 this
% sparked the development of NIFTy, a Bayesian inference library with a
% (specialized) second order minimization scheme and a custom-built AD engine on
% top of NumPy. In this talk we introduce how NIFTy realizes AD via
% linearization- and transposition-rules. Furthermore, we discuss how using AD
% for second order minimization affects the choice of rematerialization
% strategies. Attendees will learn core concepts for building their own simple
% AD framework and why linearizations and transpositions are highly desirable
% for efficient second order minimization.

\section{AD in Astrophysical Imaging}  % Background
\frame[plain,noframenumbering]{\vfill\centering\tableofcontents[sectionstyle=show/shaded,subsectionstyle=show/hide]\vfill}

\subsection{What}  % Introduction, Background
\begin{frame}
	\frametitle{\insertsection}
	\framesubtitle{\insertsubsection}

	TODO: picture of 3D dust here
\end{frame}

\begin{frame}
	\frametitle{\insertsection}
	\framesubtitle{\insertsubsection}

	\begin{itemize}
		\item TODO: Big Astro stuff
		\item TODO: eventually optimize a cost function
		\item TODO: give example cost function
		\item TODO: need AD to do so efficiently
	\end{itemize}
\end{frame}

\subsection{Problems}  % Problem, Goal
\begin{frame}
	\frametitle{\insertsection}
	\framesubtitle{\insertsubsection}

	\begin{itemize}
		\item TODO: state examplary cost function again
		\item TODO: until ``recently'' no AD in python
		\item TODO: weird 2nd order minimization and **not just gradients**
		\item TODO: our kind of optimization (approx 2nd order)
		\item TODO: state why weird!!! (cite Fisher metric paper)
		\item TODO: give numbers of \#Hessian calls, \#grad calls
	\end{itemize}

	TODO: Solution: DIY AD -> NIFTy

	\note{* NIFTy custom AD for fast Fisher metric application}
	\note{* rest of the talk will focus on how we do this and what makes it efficient for our kind of problems}
	% \footcitetext{Selig2013,Steiniger2019}
\end{frame}

\section{Linearizations on top of NumPy}  % Methodology
\frame[plain,noframenumbering]{\vfill\centering\tableofcontents[sectionstyle=show/shaded,subsectionstyle=show/hide]\vfill}

\subsection{Example}
\begin{frame}
	\frametitle{\insertsection}
	\framesubtitle{\insertsubsection}

	TODO: Introduce dust example in 3D (?): exp -> weighting + LOS -> reduction
\end{frame}

\subsection{Idea}
\begin{frame}
	\frametitle{\insertsection}
	\framesubtitle{\insertsubsection}

	TODO: Let's wrap NumPy
	TODO: what if just could ducktape numpy-like operators together + chain rule
\end{frame}

\begin{frame}
	\frametitle{\insertsection}
	\framesubtitle{\insertsubsection}

	TODO: briefly recap forward AD
	TODO: briefly recap backwards AD
	TODO: transposed of backwards AD for efficient forward AD

	\note{* non-linearities are made much cheaper here; imagine e.g. and exp in the forward model}
\end{frame}

\begin{frame}
	\frametitle{\insertsection}
	\framesubtitle{\insertsubsection}

	TODO: scheme on example cost function: build linearizations of all components
	TODO: chain linearizations for forward pass
	TODO: transpose linearizations for backwards pass

	\note{* this has been rediscovered over and over; there is a dedicated JAX paper on this...}
\end{frame}

\subsection{Linearizations}
\begin{frame}
	\frametitle{\insertsection}
	\framesubtitle{\insertsubsection}

	TODO: show abstract class?
	TODO: need to define what happens if jvp
	TODO: needs to define what happens if vjp
\end{frame}

\begin{frame}
	\frametitle{\insertsection}
	\framesubtitle{\insertsubsection}

	TODO: demo stupid class wrapping
\end{frame}

\begin{frame}
	\frametitle{\insertsection}
	\framesubtitle{\insertsubsection}

	TODO: remember original problem of applying Fisher (without needing gradient or function)
	TODO: linearization differ from JVP/VJP in that they do not evaluate the original function
\end{frame}

\begin{frame}
	\frametitle{\insertsection}
	\framesubtitle{\insertsubsection}

	TODO: get back to original demo of chained cost function
	TODO: get linearizations by putting linearization into model and successively amending it
\end{frame}

\subsection{Linearizations for minimization objective}
\begin{frame}
	\frametitle{\insertsection}
	\framesubtitle{\insertsubsection}

	TODO: show demo in action

	\note{* congrats, you now understand the DIY AD in NIFTy}
\end{frame}

\subsection{Checkpointing}
\begin{frame}
	\frametitle{\insertsection}
	\framesubtitle{\insertsubsection}

	TODO: checkpointing is crucial as \#parameter and constants in model is huge
	TODO: checkpointing is done DIY and as efficient as desired
	TODO: no duplicate memory allocations for transposition

	\note{* looking at you JAX for excessive memory allocation for transpositions}
\end{frame}

\begin{frame}
	\frametitle{\insertsection}
	\framesubtitle{\insertsubsection}

	TODO: demo example of rematerialization with weighted sum
\end{frame}

\section{Numerical Information Field Theory}  % Result
\frame[plain,noframenumbering]{\vfill\centering\tableofcontents[sectionstyle=show/shaded,subsectionstyle=show/hide]\vfill}

\subsection{Linearizations in NIFTy}
\begin{frame}
	\frametitle{\insertsection}
	\framesubtitle{\insertsubsection}

	TODO: fast if individal computations are expensive and done efficiently outside of python
	TODO: describe typical model: 10M+ parameters + FFT + exp dominates thus doing poor man's AD is fine
\end{frame}

\subsection{Advantages of NIFTY-like Linearizations}
\begin{frame}
	\frametitle{\insertsection}
	\framesubtitle{\insertsubsection}

	TODO: convenience of python and speed
	TODO: custom code
	TODO: make bottle necks go brrr with C++
\end{frame}

\end{document}
